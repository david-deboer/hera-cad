\documentclass[11pt]{article}
%\usepackage{abbrevs}
\usepackage{natbib}
\usepackage{hyperref}
\usepackage{float}
\usepackage[pdftex]{graphicx}     % could insert ``draft'' between []
\pagestyle{empty}

\setlength{\oddsidemargin}{0pt} % there is 1 inch before the
                                % side margin in ``article'' class
\setlength{\textwidth}{6.5in}

\setlength{\voffset}{0pt}
\setlength{\topmargin}{-0.75in}     % there is 1 inch before the
                                % top margin in ``article'' class and
                                % then room for header, etc.
\setlength{\textheight}{10.0in}
%%%%%%%%%%%

\newcommand{\inch}{$^{\prime\prime}$}
\newcommand{\foot}{$^{\prime}$}
\renewcommand{\deg}{^\circ}

\begin{document}
\title{HERA Installation Procedure}
\author{David DeBoer}
\maketitle

\setcounter{tocdepth}{3}
\tableofcontents

\section{Pole Installation}
After surveying the entire plat, the poles are installed on center with a $\pm$10cm tolerance, per Figure \ref{fig:poles}.  The number of rows and poles per row is set by the hex order.  Table \ref{tab:hexnums} shows important numbers for different orders.
\begin{figure}[H]
\centering
\includegraphics[width=\textwidth]{graphics/bigPoles.png}
\label{fig:poles}
\caption{Layout of poles.  Dimensions in meters.}
\end{figure}

\begin{table}[H]
\centering
\caption{Numbers}.
\begin{tabular}{| l | l |} \hline
order & total number \\ \hline
4 & x \\ \hline
\end{tabular}
\label{tab:hexnums}
\end{table}

\section{Pole Height}
Each trio of poles supports an antenna.  A fiducial constant height must be measured for each trio and marked with a small eye-bolt.  From within the circle circumscribed by the poles, this height may be measured by a theodolite along a bubble-level line or by a total station at constant $z$.  This height should be about 1.5-meters above the ground.
\begin{figure}[H]
\includegraphics[width=\textwidth]{graphics/poles_and_ants.png}
\label{fig:poles_ants}
\caption{Layout of poles.  Dimensions in meters.}
\end{figure}

\section{Hub Centering and Leveling}
With each pole with a constant height eye-bolt, the following is used to center the hub center:
\subsection{Radial lines}
A arrangement of three fixed length lines and three match spring attached on a small inner ring with a plumb-bob is used. See Figure refPicture.

\subsection{Jig}
With the center point defined by the radial lines and hanging bob, a jig as described below is used.


\bibliographystyle{plainnat}
\bibliography{YOUR BIBLIOGRAPHY HERE}
\end{document}
