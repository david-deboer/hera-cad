\documentclass[11pt]{article}
%\usepackage{abbrevs}
\usepackage{natbib}
\usepackage{hyperref}
\usepackage{float}
\usepackage[pdftex]{graphicx}     % could insert ``draft'' between []
\usepackage{stmaryrd}
\usepackage{appendix}
\pagestyle{empty}

\setlength{\oddsidemargin}{0pt} % there is 1 inch before the
                                % side margin in ``article'' class
\setlength{\textwidth}{6.5in}

\setlength{\voffset}{0pt}
\setlength{\topmargin}{-0.75in}     % there is 1 inch before the
                                % top margin in ``article'' class and
                                % then room for header, etc.
\setlength{\textheight}{10.0in}
%%%%%%%%%%%

\newcommand{\inch}{$^{\prime\prime}$}
\newcommand{\foot}{$^{\prime}$}
\renewcommand{\deg}{^\circ}

\begin{document}
\title{HERA Installation Procedure}
\author{David DeBoer}
\maketitle

\setcounter{tocdepth}{3}
\setcounter{section}{0}
\tableofcontents

\section{Overview}
HERA element installation proceeds along the following broad lines:

\begin{enumerate}
\item Setting the poles
\item Setting the hub
\item Setting the rim
\item Setting the spars
\item Setting the surface
\item Setting the feed
\end{enumerate}

\section{Setting the Poles}
After surveying the entire plat, the poles are installed on center with a $\pm$10cm tolerance from the surveyed locations.  The procurement and installation of the poles will be done by contractors.  The specifications are outlined in a separate document.

Each trio of poles supports an antenna and each interior pole is shared by three antennas.  A small eye-bolt at a fiducial height must be installed for each trio according to Procedure P-1.

\vspace{0.5cm}
\underline{\textbf{Procedure P-1}}
\begin{enumerate}
\item identify the three poles to be used by the element
\item set up the instrument at the approximate center (total station, theodolite, laser level)
\item measure a fixed height approximately 1.5m above ground level (set to horizontal level, or fixed $z$ with total station)
\item install a small eye-bolt at the closest radial point from the center and pointed to the center
\end{enumerate}

\section{Setting the Hub}
Using the centering jig and the installed eye-bolts, the following pieces are used as the hub and to center the hub.  Figure \ref{fig:centeredHub} shows the hub as set prior to being filled with concrete and Figure \ref{fig:centeredHubDrawing} shows a drawing of the South African versions (albeit with some redesign of the hub jig).

\begin{enumerate}
\item hub outer ring:  this consists of two pieces of approximately 250mm $\times$ 1600mm pieces rolled to a 1m diameter and riveted together to form a circle.
\item hub inner ring:  this consists of one piece of approximately 250mm $\times$ 1500mm rolled to a 0.5m diameter and riveted to itself to form a circle.
\item support spars:  qty 12 ???64mm??? by ???3m??? PVC pipes. (OD matters here, to fit in jig and mate with other pieces).
\item spar sleeves:  qty 12 ???75mm??? by 300mm PVC pipes.  (ID matters here to get the correct angle.  OD matters to fit into jig.)
\item centering jig: consists of three fixed length lines and three match spring attached on a small inner ring with a plumb-bob.
\item hub jig holds the two rings to be concentric and provides surfaces and support to center under the centering jig and level
\item ???700mm??? length of ???1cm??? rebar rod
\item nails:  qty 24, inserted into holes in the supports and sleeves to help constrain them.
\end{enumerate}

\begin{figure}[H]
\centering
\includegraphics[width=0.8\textwidth]{graphics/hubCentering.jpg}
\caption{This figure shows the pieces as listed as implemented at Green Bank.  Note that in the US, the hub rings are cardboard tube and the hub jig is wood.}
\label{fig:centeredHub}
\end{figure}

\begin{figure}[H]
\centering
\includegraphics[width=1.0\textwidth]{jigs/hubRingJig_withHub.pdf}
\caption{This drawing shows the pieces as listed for South Africa, note that the sleeves and spars are not shown here, as they would normally be in this configuration.}
\label{fig:centeredHubDrawing}
\end{figure}

The following procedure is used:

\begin{enumerate}
\item a ???2mm??? hole is drilled through the support spars and sleeves at a distance of ???80mm??? from the end.
\item the centering jig is clipped onto the three eye-bolts, then onto the center ring to define the actual center of the element.  A plumb-bob hangs down to accurately place the hub.
\item the two rings are concentrically placed within the hub jig and the support spars and sleeves are inserted
\item the constraining nails are inserted into the PVC and arranged to be flush with the inner ring (within the concentric ring)
\item the rebar is inserted through the small holes at the top of the inner ring and centered
\item the assembly is accurately (within 1cm) centered underneath the centering jig and leveled in perpendicular directions
\item the interior of the concentric ring (the ``doughnut'') is filled to the top with cement and allowed to set
\end{enumerate}

\newpage
\appendix
\section{Checklist}

\renewcommand{\labelitemi}{$\boxempty$}
\renewcommand{\labelitemii}{$\boxempty$}
\begin{itemize}
\item Install poles
\item Survey pole marker fiducial heights ($\approx$ 1.5m) and install eye-bolts
\item Install centering jig (H7jcen)
\item Install hub forms with hub jig (H7jhub), using H7jcen to center
\item Rotate hub to point to poles and conduit exit to correct direction
\item Level with H7jhub and stakes
\item Install support spars (P/N) (long 50mm PVC in bottom holes); peg in place with nails 
\item Install surface spar sleeves (P/N) (short 65mm PVC in top holes); peg in place with nails
\item Install circular rebar (P/N) and straight central rebar (P/N)
\item Install exit conduit (P/N)
\item Pour concrete to top of form and let set
\item Mark center point on rebar with cable tie (this is the vertex) and remove jigs
\item Set-up Total Station on center and 60in above rebar
\item Survey and install pole vertical assemblies (H1p2v)
	\begin{itemize}
	\item affix prism target on end of horizontal spar
	\item bottom at z=14.0in
	\item shim to vertical
	\item use qty 3 3/8'' lag bolts
	\end{itemize}
\item Survey and install pole horizontal assembly (H1p2h)
	\begin{itemize}
	\item target at angle = 8.08$^\circ$, distance = 274.85in (z = 38.63in)
	\end{itemize}
\item Rough in posts (H1pxxx) between the poles using rim pieces (H1vvvvv), including post forms (   )
\item If unshared post or first installation of a shared post, survey and brace
	\begin{itemize}
	\item target at angle = 8.08$^\circ$, distance = 274.85in (z=38.63in)
	\end{itemize}
\item If shared post, survey using the offset piece (H1ggggg) and brace:  target at ...
\item Install rebar and concrete,  let set and remove bracing
\item Install vertical pvc support pieces on end of support spars (should point straight up) and level 
\item Install parabolic spars in hub (be sure pre-drilled holes are facing up) and affix at ends
\item Install pvc cross pieces ( )
\item Install cross piece spars and affix at ends
\item Install hub access platform at appropriate location
\item Install all wire cloth panel A pieces
\item Install all but access section of panel B pieces (with furring strips on outer), sandwiching A-B-cross
\item Install all but access section of panel C pieces (with furring strips on outer), sandwiching B-C-furring
\item Install all panel D pieces, sandwiching C-D-furring
\item Install all panel E pieces
\item install door pieces
\end{itemize}

\bibliographystyle{plainnat}
\bibliography{YOUR BIBLIOGRAPHY HERE}
\end{document}
